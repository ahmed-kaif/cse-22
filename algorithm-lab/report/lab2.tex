\section{Convex Hull}
\subsection{Problem Statement}
\begin{itemize}
    \item Implement Quickhull and Graham Scan algorithm to find the convex hull of 2D points. 
    \item Visualize the input 2D points and the convex hull boundary. 
    \item Compare Quickhull and Graham Scan based on various input size on randomly
        generated points.  The comparison metric should be the execution time of 
        each sorting algorithm.
\end{itemize}
\subsection{Code}
\begin{code}
    \caption{quick\_hull.cpp}
    \cppcode{../quickhull.cpp}
    \label{code:quickhull}
\end{code}

\begin{code}
    \caption{graham\_scan.cpp}
    \cppcode{../graham_scan.cpp}
    \label{code:graham}
\end{code}
\begin{code}
    \caption{Makefile}
\begin{minted}[
linenos=true,
fontfamily=tt,
fontsize=\small,
linenos=true,
numberblanklines=true,
breaklines=true,
numbersep=5pt,
gobble=0,
frame=leftline,
framerule=0.4pt,
framesep=2mm,
funcnamehighlighting=true,
tabsize=4,
obeytabs=false,
mathescape=false
samepage=false, %with this setting you can force the list to appear on the same page
showspaces=false,
showtabs =false,
texcl=false
    ]{make}
CC=g++

convex: points graham quick

points: generate_points.cpp
	$(CC) $^ -o points.out
	./points.out
graham:	graham_scan.cpp
	$(CC) $^ -o graham_scan.out
	./graham_scan.out

quick: quickhull.cpp
	$(CC) $^ -o quickhull.out
	./quickhull.out

clean:
	rm *.out
    \end{minted}
\end{code}


\begin{code}
    \caption{visualization.py}
    \begin{minted}[
linenos=true,
fontfamily=tt,
fontsize=\small,
linenos=true,
numberblanklines=true,
breaklines=true,
numbersep=5pt,
gobble=0,
frame=leftline,
framerule=0.4pt,
framesep=2mm,
funcnamehighlighting=true,
tabsize=4,
obeytabs=false,
mathescape=false
samepage=false, %with this setting you can force the list to appear on the same page
showspaces=false,
showtabs =false,
texcl=false
    ]{python}
import matplotlib.pyplot as plt 
%matplotlib inline
X = [ 44, 63, 90, 43, 54, 26, 42, 47, 57, 2, 61, 72, 24, 88, 82, 78, 33, 74, 55, 19, 99, 24, 42, 73, 18, 32, 41, 43, 64, 49, 8, 73, 66, 13, 66, 32, 27, 8, 82, 69, 5, 80, 59, 12, 56, 70, 86, 7, 40, 74, 54, 20, 65, 51, 59, 96, 76, 60, 100, 60, 83, 75, 23, 22, 4, 18, 57, 89, 16, 18, 11, 90, 43, 71, 24, 1, 11, 78, 60, 46, 51, 72, 51, 79, 100, 93, 12, 99, 82, 47, 51, 64, 26, 97, 92, 100, 56, 100, 31, 24 ]
Y = [ 55, 98, 64, 46, 32, 64, 98, 29, 44, 83, 16, 14, 57, 82, 26, 77, 40, 22, 68, 61, 44, 93, 23, 50, 74, 30, 55, 16, 83, 97, 26, 92, 46, 72, 31, 64, 8, 20, 80, 99, 53, 97, 74, 74, 60, 16, 42, 3, 72, 5, 58, 80, 28, 46, 72, 64, 27, 34, 24, 8, 29, 20, 33, 62, 48, 58, 37, 21, 40, 75, 65, 86, 49, 94, 30, 7, 27, 33, 52, 63, 63, 7, 61, 67, 96, 62, 82, 54, 69, 6, 100, 13, 41, 85, 42, 42, 71, 6, 78, 82 ]
x = [ 100, 69, 51, 42, 24, 2, 1, 7, 74, 100, 100 ]
y = [ 96, 99, 100, 98, 93, 83, 7, 3, 5, 6, 96]

plt.scatter(X,Y, label="n=100")
plt.plot(x,y,color="red")
plt.xlabel("x") 
plt.ylabel("y")
plt.legend()
plt.savefig("convex_100pt.png", dpi=300, bbox_inches="tight")
    \end{minted}
\end{code}

\subsection{Output}

\begin{figure}[H]
    \centering
    \begin{subfigure}[b]{0.4\textwidth}
        \centering
        \includegraphics[width=\textwidth]{./img/lab2/p5k.png}
        \caption{Execution time for n=5000}
    \end{subfigure}
    \hfill
    \begin{subfigure}[b]{0.4\textwidth}
        \centering
        \includegraphics[width=\textwidth]{./img/lab2/p10k.png}
        \caption{Execution time for n=10000}
    \end{subfigure}
    \hfill
    \begin{subfigure}[b]{0.4\textwidth}
        \centering
        \includegraphics[width=\textwidth]{./img/lab2/p50k.png}
        \caption{Execution time for n=50000}
    \end{subfigure}
    \hfill
    \begin{subfigure}[b]{0.4\textwidth}
        \centering
        \includegraphics[width=\textwidth]{./img/lab2/p1lack.png}
        \caption{Execution time for n=100000}
    \end{subfigure}
    
    \caption{Convex hull algorithm execution time}
    \label{fig:task1}

\end{figure}

\subsubsection*{Summary of Execution time}
\begin{table}[H]
    \centering
    \caption{Comparison table for Graham Scan \& Quick Hull algorithm execution time}
    \label{tab:comp2}
    \begin{tabular}{lSS}
        \toprule
        \multicolumn{1}{c}{Input Size} & \multicolumn{1}{c}{Graham Scan (ms)} & \multicolumn{1}{c}{Quick Hull (ms)} \\
        \midrule
        1,000   & 0.460258 & 0.413883 \\
        5,000   & 2.735560  & 1.201980 \\
        10,000  & 5.525400   & 2.739060  \\
        50,000  & 29.784500  & 8.964970 \\
        100,000 & 61.406500  & 17.945400  \\
        \bottomrule
    \end{tabular}
\end{table}
\subsubsection*{Visualization}

\begin{figure}[H]
    \centering
    \begin{subfigure}[b]{0.5\textwidth}
        \centering
        \includegraphics[width=\textwidth]{./img/lab2/convex_100pt.png}
        \caption{Convex hull visualization using graham scan for 100 points}
    \end{subfigure}
    \hfill
    \begin{subfigure}[b]{0.5\textwidth}
        \centering
        \includegraphics[width=\textwidth]{./img/lab2/convex_1kpt.png}
        \caption{Convex hull visualization using quick hull for 1000 points}
    \end{subfigure}
    
    \caption{Convex hull Visualization}
    \label{fig:task2}
\end{figure}
\subsection{Analysis \& Discussion}
Graham Scan is faster than quick hull for very number of points. 
But with increasing number of points quick hull is significantly efficient
than graham scan. 
This is because Graham scan needs to sort the points before doing its
calculation. And this is more costly. 
In case of quick hull, there is no need to sort the points, rather it can find 
the points by divide and conquer approach, which significantly decreases 
the execution time for quick hull.
